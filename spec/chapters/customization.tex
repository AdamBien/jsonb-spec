\chapter{Customizing Mapping}
\label{customization}

This section defines several ways how to customize the default behavior. The default behavior can be customized annotating a given field, JavaBean property, type or package, or by providing an implementation of particular strategy, e.g. PropertyOrderStrategy. JSON Binding provider MUST support these customization options.[JSB-\ref{customization}-1]

\section{Customizing Property Names}
\label{sec:custom_property_names}

There are two standard ways how to customize serialization of field (or JavaBean property) to JSON document. The same applies to deserialization. The first way is to annotate field (or JavaBean property) with javax.json.bind.annotation.JsonbProperty annotation. The second option is to set javax.json.bind.config.PropertyNamingPolicy.

\subsection{javax.json.bind.annotation.JsonbTransient}
\label{subsec:JsonbTransient}

JSON Binding implementations MUST NOT serialize fields,  JavaBean properties or types annotated with javax.json.bind.annotation.JsonbTransient. When placed on a class, indicates that the class shouldn't be mapped to JSON by itself. Properties on such class will be mapped to JSON along with its derived classes, as if the class is inlined. [JSB-\ref{subsec:JsonbTransient}-1]

JsonbTransient annotation is mutually exclusive with all other JSON Binding defined annotations. If this condition is not met, JSON Binding implementation MUST throw JsonbException. [JSB-\ref{subsec:JsonbTransient}-2]

\subsection{javax.json.bind.annotation.JsonbProperty}
\label{subsec:JsonbProperty}

According to default mapping \ref{sec:naming}, property names are serialized unchanged to JSON document (identity transformation). To provide custom name for given field (or JavaBean property), javax.json.bind.annotation.JsonbProperty may be used. JsonbProperty annotation may be specified on field, getter or setter method. If specified on field, custom name is used both for serialization and deserialization. If javax.json.bind.annotation.JsonbProperty is specified on getter method, it is used only for serialization. If javax.json.bind.annotation.JsonbProperty is specified on setter method, it is used only for deserialization. It is possible to specify different values for getter and setter method for javax.json.bind.annotation.JsonbProperty annotation. In such case the different custom name will be used for serialization and deserialization. [JSB-\ref{subsec:JsonbProperty}-1]

\subsection{javax.json.bind.config.PropertyNamingStrategy}
\label{subsec:PropertyNamingStrategy}

To customize name translation of properties, JSON Binding provides javax.json.bind.config.PropertyNamingStrategy interface.

Interface javax.json.bind.config.PropertyNamingStrategy provides the most common property naming strategies.

\begin{list}{$\bullet$}{\parsep 0em \labelwidth 0em}
\item IDENTITY
\item LOWER\_CASE\_WITH\_DASHES
\item LOWER\_CASE\_WITH\_UNDERSCORES
\item UPPER\_CAMEL\_CASE
\item UPPER\_CAMEL\_CASE\_WITH\_SPACES
\item CASE\_INSENSITIVE
\end{list}

The detailed description of property naming strategies can be found in javadoc.

The way to set custom property naming strategy is to use javax.json.bind.JsonbConfig::withPropertyNamingStrategy method. [JSB-\ref{subsec:JsonbProperty}-1]

\subsection{Property names resolution}
\label{sec:property_names_resolution}
Property name resolution consists of two phases:

1. Standard override mechanism
2. Applying property name resolution, which involves the value of @JsonbProperty

If duplicate name is found exception MUST be thrown.
The definition of duplicate (non-unique) property can be found in \ref{sec:uniqueness_properties}. [JSB-\ref{sec:property_names_resolution}-1]


\section{Customizing Property Order}
\label{sec:custom_property_order}

To customize order of serialized properties, JSON Binding provides javax.json.bind.config.PropertyOrderStrategy class.

Class javax.json.bind.config.PropertyOrderStrategy provides the most common property order strategies.

\begin{list}{$\bullet$}{\parsep 0em \labelwidth 0em}
\item LEXICOGRAPHICAL
\item ANY
\item REVERSE
\end{list}

The detailed description of property order strategies can be found in javadoc.

The way to set custom property order strategy is to use javax.json.bind.JsonbConfig::withPropertyOrderStrategy method. [JSB-\ref{sec:custom_property_order}-2]

To customize order of serialized properties only for one specific type, JSON Binding provides javax.json.bind.annotation.JsonbPropertyOrder annotation. Order specified by JsonbPropertyOrder annotation overrides order specified by PropertyOrderStrategy. [JSB-\ref{sec:custom_property_order}-3]

The order is applied to already renamed properties as stated in \ref{sec:custom_property_names}.

\section{Customizing Null Handling}
\label{sec:custom_null_handling}

There are three ways how to change default null handling. The first option is to annotate type or package with javax.json.bind.annotation.JsonbNillable annotation. The second option is to annotate field or JavaBean property with javax.json.bind.annotation.JsonbProperty and to set nillable parameter to true. The third option is to set config wide configuration via JsonbConfig::withNullValues method.

If annotations (JsonbNillable or JsonbProperty) on different level apply to the same field (or JavaBean property) or if there is config wide configuration and some annotation (JsonbNillable or JsonbProperty) which apply to the same field (or JavaBean property), the annotation with the smallest scope applies. For example, if there is type level JsonbNillable annotation applied to some class with field which is annotated with JsonbProperty annotation with nillable = false, then JsonbProperty annotation overrides JsonbNillable annotation.

\subsection{javax.json.bind.annotation.JsonbNillable}
\label{subsec:JsonbNillable}

To customize the result of serializing field (or JavaBean property) with null value, JSON Binding provides javax.json.bind.annotation.JsonbNillable and javax.json.bind.annotation.JsonbProperty annotations.

When given object (type or package) is annotated with javax.json.bind.annotation.JsonbNillable annotation, the result of null value will be presence of associated property in JSON document with explicit null value. [JSB-\ref{subsec:JsonbNillable}-1]

The same behavior as JsonbNillable, but only at field, parameter and method (JavaBean property) level is provided by javax.json.bind.annotation.JsonbProperty annotation with its nillable parameter. [JSB-\ref{subsec:JsonbNillable}-2]

JSON Binding implementations MUST implement override of annotations according to target of the annotation (FIELD, PARAMETER, METHOD, TYPE, PACKAGE). Type level annotation overrides behavior set at the package level. Method, parameter or field level annotation overrides behavior set at the type level. [JSB-\ref{subsec:JsonbNillable}-3]

\subsection{Global null handling configuration}
\label{subsec:NullSerializationPolicy}

Null handling behavior can be customized via javax.json.bind.JsonbConfig::withNullValues method.

The way to enforce serialization of null values, is to call method javax.json.bind.JsonbConfig::withNullValues with parameter true. \\
The way to skip serialization of null values is to call method javax.json.bind.JsonbConfig::withNullValues with parameter false. [JSB-\ref{subsec:NullSerializationPolicy}-1]

\section{I-JSON support}
\label{sec:i_json}

I-JSON (short for "Internet JSON") is a restricted profile of JSON designed to maximize interoperability and increase confidence that software can process it successfully with predictable results. The profile is defined in \cite{rfc7159}.

JSON Binding provides full support for I-JSON standard. Without any configuration, JSON Binding produces JSON documents which are compliant with I-JSON with three exceptions.

\begin{list}{$\bullet$}{\parsep 0em \labelwidth 0em}
\item JSON Binding does not restrict the serialization of top-level JSON texts that are neither objects nor arrays. The restriction should happen at application level.
\item JSON Binding does not serialize binary data with base64url encoding.
\item JSON Binding does not enforce additional restrictions on dates/times/duration.
\end{list}

These exceptions refer only to recommended areas of I-JSON.

To enforce strict compliance of serialized JSON documents, JSON Binding implementations MUST implement configuration option "jsonb.i-json.strict-ser-compliance".
[JSB-\ref{sec:i_json}-1]

The way to enable strict compliance of serialized JSON documents, is to call method \\ JsonbConfig::withStrictIJSONSerializationCompliance with parameter true.

\subsection{Strict date serialization}
\label{subsec:strict_date_serialization}

Uppercase rather than lowercase letters MUST be used.  [JSB-\ref{subsec:strict_date_serialization}-1] The timezone MUST always be included and optional trailing seconds MUST be included even when their value is "00". [JSB-\ref{subsec:strict_date_serialization}-2]

JSON Binding implementations MUST serialize java.util.Date, java.util.Calendar, java.util.GregorianCalendar, java.time.LocalDate, java.time.LocalDateTime and java.time.Instant in the same format as java.time.ZonedDateTime. [JSB-\ref{subsec:strict_date_serialization}-3]

The result of serialization of duration must conform to the "duration" production in Appendix A of RFC 3339, with the same additional restrictions. [JSB-\ref{subsec:strict_date_serialization}-4]

\section{Simple values}
\label{sec:simple_values}

Using javax.json.bind.annotation.JsonbValue annotation, a class can be mapped to a simple value. Class can contain at most one mapped property or field that is annotated with javax.json.bind.annotation.JsonbValue, otherwise JsonbException MUST be thrown.
[JSB-\ref{sec:simple_values}-1]

Annotation javax.json.bind.annotation.JsonbValue indicates that result of the annotated non-void method or field or constructor parameter will be used as the single value to representing the instance. In case of non-void method annotated with JsonbValue annotation, JsonbException MUST be thrown.
[JSB-\ref{sec:simple_values}-2]

\section{Custom instantiation}
\label{sec:custom_instantiation}

In many scenarios instantiation with the use of default constructor is not enough. To support these scenarios, JSON Binding provides javax.json.bind.annotation.JsonbCreator annotation.

At most one JsonbCreator annotation can be used to annotate custom constructor or static void factory method in a class, otherwise JsonbException MUST be thrown. [JSB-\ref{sec:custom_instantiation}-1]

Factory method annotated with JsonbCreator annotation should return instance of particular class this annotation is used for, otherwise JsonbException MUST be thrown. [JSB-\ref{sec:custom_instantiation}-2]

Parameters of constructor/factory method annotated with JsonbCreator will be mapped from JSON fields with the same name. The name of a parameter can be changed annotating given parameter with JsonbProperty annotation. When a JSON field is not mappable to a parameter with the same name, JsonbException MUST be thrown. [JSB-\ref{sec:custom_instantiation}-3]

\section{Custom visibility}
\label{sec:custom_visibility}

To customize scope and field access strategy as specified in section \ref{subsec:fieldstrategy}, it is possible to specify javax.json.bind.annotation.JsonbVisibility annotation or to override default behavior globally calling JsonbConfig::withPropertyVisibilityStrategy method with given custom property visibility strategy. [JSB-\ref{sec:custom_visibility}-1]

\section{Custom mapping}
\label{sec:jsonb_adapter}

To provide custom mapping for specific java type, it is necessary to implement javax.json.bind.adapter.JsonbAdapter interface. [JSB-\ref{sec:jsonb_adapter}-1]

There are two ways how to register custom JsonbAdapter. Using JsonbConfig::withAdapters method or annotating specific field or JavaBean property with JsonbTypeAdapter annotation. [JSB-\ref{sec:jsonb_adapter}-2]

JsonbAdapter registered via JsonbConfig::withAdapters is visible to all serialize/deserialize operations performed with given JsonbConfig. JsonbAdapter registered with annotation is visible to serialize/deserialize operation used only for given field/JavaBean property annotated. [JSB-\ref{sec:jsonb_adapter}-3]

\section{Custom date format}
\label{sec:custom_date_format}

To specify custom date format, it is necessary to annotate given annotation target with \\ javax.json.bind.annotation.JsonbDateFormat annotation. [JSB-\ref{sec:custom_date_format}-1] JsonbDateFormat annotation can be applied to the following targets:

\begin{list}{$\bullet$}{\parsep 0em \labelwidth 0em}
\item field
\item method
\item type
\item parameter
\item package
\end{list}

Annotation applied to more specific target overrides the same annotation applied to target with wider scope. For example, annotation applied to type target will override the same annotation applied to package target. [JSB-\ref{sec:custom_date_format}-2]

\section{Custom number format}
\label{sec:custom_number_format}

To specify custom number format, it is necessary to annotate given annotation target with \\ javax.json.bind.annotation.JsonbNumberFormat annotation. [JSB-\ref{sec:custom_number_format}-1] JsonbNumberFormat annotation can be applied to the following targets:

\begin{list}{$\bullet$}{\parsep 0em \labelwidth 0em}
\item field
\item method
\item type
\item parameter
\item package
\end{list}

Annotation applied to more specific target overrides the same annotation applied to target with wider scope. For example, annotation applied to type target will override the same annotation applied to package target. [JSB-\ref{sec:custom_number_format}-2]

\section{Custom binary data handling}
\label{sec:custom_binary_data}

To customize encoding of binary data, JSON Binding provides javax.json.bind.config.BinaryDataStrategy class.

Class javax.json.bind.config.BinaryDataStrategy provides the most common binary data encodings.

\begin{list}{$\bullet$}{\parsep 0em \labelwidth 0em}
\item BYTE
\item BASE\_64
\item BASE\_64\_URL
\end{list}

The detailed description of binary encoding strategies can be found in javadoc.

The way to set custom binary data handling strategy is to use javax.json.bind.JsonbConfig::withBinaryDataStrategy method. [JSB-\ref{sec:custom_binary_data}-1]