\chapter{Default Mapping}
\label{defaultmapping}

This section defines the default binding (representation) of Java components and classes within Java programming language to JSON documents. The default binding defined here can be further customized as specified in Chapter \ref{customization} - \nameref{customization}.

\section{General}
\label{sec:general}
JSON Binding implementations ('implementations' in further text) MUST support binding of JSON documents as defined in RFC 7159 JSON Grammar \cite{rfc7159}. Marshalled JSON output MUST conform to the RFC 7159 JSON Grammar \cite{rfc7159} and be encoded in UTF-8 encoding as defined in Section 8.1 (Character Encoding) of RFC 7159 \cite{rfc7159}. [JSB-\ref{sec:general}-1] 
Implementations MUST support unmarshalling of documents conforming to RFC 7159 JSON Grammar \cite{rfc7159}. [JSB-\ref{sec:general}-2] In addition, implementations SHOULD NOT allow unmarshalling of RFC 7159 \cite{rfc7159} non-conforming text (e.g. unsupported encoding, ...) and report error in such case. [JSB-\ref{sec:general}-3] Detection of UTF encoding of unmarshalled document is done as defined in the Section 3 (Encoding) of RFC 4627 \cite{rfc4627}. [JSB-\ref{sec:general}-4] Implementations SHOULD ignore presence of UTF byte order mark (BOM) and not treat it as an error.[JSB-\ref{sec:general}-5]

\section{Errors}
\label{sec:errors}
Implementations SHOULD NOT allow unmarshalling of RFC 7159 \cite{rfc7159} non-conforming text (e.g. unsupported encoding, ...) and report error in such case. [JSB-\ref{sec:errors}-1] Implementation should report error also during unmarshalling operation, if it is not possible to represent JSON document value in the expected Java type. [JSB-\ref{sec:errors}-2]

\section{Basic Java Types}
\label{sec:basic}
Implementations MUST support binding of the following basic Java classes and their corresponding primitive types: [JSB-\ref{sec:basic}-1]
\begin{list}{$\bullet$}{\parsep 0em \labelwidth 0em}
\item java.lang.String
\item java.lang.Character
\item java.lang.Byte
\item java.lang.Short
\item java.lang.Integer
\item java.lang.Long
\item java.lang.Float
\item java.lang.Double
\item java.lang.Boolean
\end{list}

\subsection{java.lang.String, Character}
\label{subsec:string}
Instances of type java.lang.String and java.lang.Character are marshalled to JSON String values as defined within RFC 7159 Section 7 (Strings) \cite{rfc7159} in UTF-8 encoding without byte order mark. [JSB-\ref{subsec:string}-1] Implementations SHOULD support unmarshaling of JSON text in other (than UTF-8) UTF encodings into java.lang.String instances.[JSB-\ref{subsec:string}-2]

\subsection{java.lang.Byte, Short, Integer, Long, Float, Double}
\label{subsec:number}
Instances of type java.lang.Byte, Short, Integer, Long, Float, Double and their corresponding primitive types are marshalled to JSON Number with conversion defined in specification for their corresponding toString method [JSB-\ref{subsec:number}-1]. Unmarshalling of JSON value into java.lang.Byte, Short, Integer, Long, Float, Double instance or corresponding primitive type is done with conversion as defined in the specification for their corresponding
parse\${Type} method, such as java.lang.Byte.parseByte for Byte. [JSB-\ref{subsec:number}-2]

\subsection{java.lang.Boolean}
\label{subsec:boolean}
Instances of type java.lang.Boolean and its corresponding boolean primitive type are marshalled to JSON value with conversion defined in specification for java.lang.Boolean.toString method [JSB-\ref{subsec:boolean}-1]. Unmarshalling of JSON value into java.lang.Boolean instance or boolean primitive type is done with conversion as defined in specification for java.lang.Boolean.parseBoolean method. [JSB-\ref{subsec:boolean}-2]

\subsection{java.lang.Number}
\label{subsec:abstractnumber}
Instances of type java.lang.Number (if their more concrete type is not defined elsewhere in this chapter) are marshalled to JSON string by retrieving double value returned from java.lang.Number.doubleValue() method and converting the value to JSON Number as defined in \fullref{subsec:number}. [JSB-\ref{subsec:abstractnumber}-1].\\
Unmarshalling of JSON value into Java type java.lang.Number should return instance of java.math.BigDecimal by using conversion as defined in the specification for constructor of java.math.BigDecimal with java.lang.String. [JSB-\ref{subsec:abstractnumber}-2]

\section{Specific Standard Java SE Types}
\label{sec:specific}
Implementations MUST support binding of the following standard Java SE classes: [JSB-\ref{sec:specific}-1]
\begin{list}{$\bullet$}{\parsep 0em \labelwidth 0em}
\item java.math.BigInteger
\item java.math.BigDecimal
\item java.net.URL
\item java.net.URI
\item java.util.Optional
\item java.util.OptionalInt
\item java.util.OptionalLong
\item java.util.OptionalDouble
\end{list}

\subsection{java.math.BigInteger, BigDecimal}
\label{subsec:bignumber}
Instances of type java.math.BigInteger, BigDecimal are marshalled to JSON Number with conversion defined in specification for their toString method [JSB-\ref{subsec:bignumber}-1]. Unmarshalling of JSON value into java.math.BigInteger, BigDecimal instance is done with conversion as defined in the specification for constructor of java.math.BigInteger, BigDecimal with java.lang.String. [JSB-\ref{subsec:bignumber}-2]

\subsection{java.net.URL, URI}
\label{subsec:url}
Instances of type java.net.URL, URI are marshalled to JSON String value with conversion defined in specification for their toString method [JSB-\ref{subsec:url}-1]. Unmarshalling of JSON value into java.net.URL, URI instance is done with conversion as defined in the specification for constructor of java.net.URL, URI with java.lang.String input. [JSB-\ref{subsec:url}-2]

\subsection{java.util.Optional, OptionalInt, OptionalLong, OptionalDouble}
\label{subsec:optional}
Non-empty instances of type java.util.Optional, OptionalInt, OptionalLong, OptionalDouble are marshalled to JSON value by retrieving their contained instance and converting it to JSON value based on its type and corresponding mapping definitions within this chapter. [JSB-\ref{subsec:optional}-1] Empty optional instances marshalled as object instance properties are ignored in marshalling. [JSB-\ref{subsec:optional}-2] Empty optional instances marshalled as JSON array elements are marshalled as null value [JSB-\ref{subsec:optional}-3]. Unmarshalling into Optional, OptionalInt, OptionalLong, OptionalDouble returns empty optional value for properties which are not present in JSON document or contain null value. [JSB-\ref{subsec:url}-4] Otherwise any non-empty Optional, OptionalInt, OptionalLong, OptionalDouble value is constructed of type unmarshalled based on mappings defined in this chapter.[JSB-\ref{subsec:url}-5]

\section{Dates}
\label{sec:dates}
Implementations MUST support binding of the following standard Java date/time classes: 
[JSB-\ref{sec:dates}-1]
\begin{list}{$\bullet$}{\parsep 0em \labelwidth 0em}
\item java.util.Date
\item java.util.Calendar
\item java.util.GregorianCalendar
\item java.util.TimeZone
\item java.util.SimpleTimeZone
\item java.time.Instant
\item java.time.Duration
\item java.time.Period
\item java.time.LocalDate
\item java.time.LocalTime
\item java.time.LocalDateTime
\item java.time.ZonedDateTime
\item java.time.ZoneId
\item java.time.ZoneOffset
\item java.time.OffsetDateTime
\item java.time.OffsetTime
\end{list}

If not specified otherwise in this section, GMT standard time zone and offset specified from UTC Greenwich is used. [JSB-\ref{sec:dates}-2] 
If not specified otherwise, date time format for marshalling and unmarshalling is ISO 8601 without offset, as specified in java.time.format.DateTimeFormatter.ISO\verb|_|DATE. [JSB-\ref{sec:dates}-3] 
Implementations MUST report error if the datetime string in JSON document does not correspond to the expected datetime format. [JSB-\ref{sec:dates}-4]

\subsection{java.util.Date, Calendar, GregorianCalendar}
\label{subsec:datecalendar}
Marshalling format of java.util.Date, Calendar, GregorianCalendar instances with no time information is ISO\verb|_|DATE. [JSB-\ref{subsec:datecalendar}-1]. If time information is present, the format is ISO\verb|_|DATE\verb|_|TIME [JSB-\ref{subsec:datecalendar}-2].

Implementations MUST support unmarshalling of both ISO\verb|_|DATE and ISO\verb|_|DATE\verb|_|TIME into java.util.Date, Calendar, GregorianCalendar instances. [JSB-\ref{subsec:datecalendar}-3] For properties and fields specified as java.util.Calendar type, instance of java.util.GregorianCalendar SHOULD be returned. [JSB-\ref{subsec:datecalendar}-4]

\subsection{java.util.TimeZone, SimpleTimeZone}
\label{subsec:dateTimezone}

Implementations MUST support unmarshalling of any time zone format specified in java.util.TimeZone into field or property of type java.util.TimeZone, SimpleTimezone. [JSB-\ref{subsec:dateTimezone}-1] Implementations MUST report error for deprecated three-letter time zone IDs as specified in java.util.Timezone. [JSB-\ref{subsec:dateTimezone}-2] For properties and fields specified as java.util.TimeZone, instance of java.util.SimpleTimeZone SHOULD be returned. [JSB-\ref{subsec:dateTimezone}-3]
Marshalling format of java.util.TimeZone, SimpleTimeZone is NormalizedCustomID as specified in java.util.TimeZone. [JSB-\ref{subsec:dateTimezone}-4]

\subsection{java.time.*}
\label{subsec:time}

Marshalling output for java.time.Instant instance is ISO\verb|_|INSTANT format, as specified in java.time.format.DateTimeFormatter. [JSB-\ref{subsec:time}-1]
Implementations MUST support unmarshalling of ISO\verb|_|INSTANT format from JSON string to a java.time.Instant instance. [JSB-\ref{subsec:time}-2]

Analogically, for other java.time.* classes, following mapping table matches Java types and corresponding formats: [JSB-\ref{subsec:time}-3]

\begin{tabularx}{\textwidth}{ |X|X| }
\hline
Java Type & Format \\ 
\hline
java.time.Instant & ISO\verb|_|INSTANT\\
java.time.LocalDate & ISO\verb|_|LOCAL\verb|_|DATE\\
java.time.LocalTime & ISO\verb|_|LOCAL\verb|_|TIME\\
java.time.LocalDateTime & ISO\verb|_|LOCAL\verb|_|DATE\verb|_|TIME\\
java.time.ZonedDateTime & ISO\verb|_|ZONED\verb|_|DATE\verb|_|TIME\\
java.time.OffsetDateTime & ISO\verb|_|OFFSET\verb|_|DATE\verb|_|TIME\\
java.time.OffsetTime & ISO\verb|_|OFFSET\verb|_|TIME\\
\hline
\end{tabularx}

Implementations MUST support unmarshalling of any time zone ID format specified in java.time.ZoneId into field or property of type java.time.ZoneId. [JSB-\ref{subsec:time}-4]
Marshalling format of java.time.ZoneId is normalized zone ID as specified in java.time.ZoneId. [JSB-\ref{subsec:time}-5]

Implementations MUST support unmarshalling of any time zone ID format specified in java.time.ZoneOffset into field or property of type java.time.ZoneOffset. [JSB-\ref{subsec:time}-6]
Marshalling format of java.time.ZoneOffset is normalized zone ID as specified in java.time.ZoneOffset. [JSB-\ref{subsec:time}-7]

Implementations MUST support unmarshalling of any duration format specified in java.time.Duration into field or property of type java.time.Duration. [JSB-\ref{subsec:time}-8] This is super-set of ISO 8601 duration format.
Marshalling format of java.time.Duration is ISO 8601 seconds based representation, such as
\verb|PT8H6M12.345S|. [JSB-\ref{subsec:time}-9]

Implementations MUST support unmarshalling of any period format specified in java.time.Period into field or property of type java.time.Period. [JSB-\ref{subsec:time}-10] This is super-set of ISO 8601 period format.
Marshalling format of java.time.Period is ISO 8601 period representation. [JSB-\ref{subsec:time}-11] Zero length period is represented as zero days \verb|'P0D'|. [JSB-\ref{subsec:time}-12]


\section{Untyped mapping}
\label{sec:untyped}
For unspecified output type of unmarshal operation, as well as where output type is specified as Object.class, implementations should unmarshal JSON document using Java runtime types specified in table below: [JSB-\ref{sec:untyped}-1]

\begin{tabularx}{\textwidth}{ |X|X| }
\hline
JSON value & Java type \\ 
\hline
object & java.util.LinkedHashMap \textless String,Object \textgreater\\
array & java.util.ArrayList \textless Object \textgreater \\
string & java.lang.String \\
number & java.math.Integer|Long|BigDecimal \\
true, false & java.lang.Boolean \\
null & null \\
\hline
\end{tabularx}

JSON number values are unmarshalled into smallest of types Integer, Long, BigDecimal which can hold the value represented by number without loss of value or precision.[JSB-\ref{sec:untyped}-2]

\section{Java Class}
\label{sec:class}
TODO - define class marshalling/unmarshalling algorithm.

\subsection{Default scope}
\label{subsec:scopedefault}

\subsection{Field access strategy}
\label{subsec:fieldstrategy}
For unmarshalling operation for a Java property, if a matching setter method exists, the method is called to set the value of the property, otherwise direct field assignment is used. [JSB-\ref{subsec:fieldstrategy}-1] For marshalling operation, if a matching getter method exists, the method is called to obtain value of the property, otherwise the value is obtained directly from the field. [JSB-\ref{subsec:fieldstrategy}-2]

\subsection{Nested Classes}
\label{subsec:nestedclass}

\subsection{Static Nested Classes}
\label{subsec:staticnested}

\subsection{Anonymous Classes}
\label{subsec:anonymous}

\section{Enum}
\label{sec:enum}
Enum instances are marshalled to JSON String value with conversion defined in specification for their toString method [JSB-\ref{sec:enum}-1]. Unmarshalling of JSON value into enum instance is done by calling enum's valueOf(String) method. [JSB-\ref{sec:enum}-2]

\section{Interfaces}
\label{sec:interfaces}
Implementations MUST support unmarshalling of specific interfaces defined in  \fullref{sec:collections}, and \fullref{subsec:abstractnumber}. [JSB-\ref{sec:interfaces}-1] Unmarshalling to other interfaces is not supported and implementations SHOULD report error in such case. [JSB-\ref{sec:interfaces}-2] If class property is defined with an interface, and not concrete type, mapping for marshalling the property is resolved based on its runtime type.[JSB-\ref{sec:interfaces}-3]

\section{Collections}
\label{sec:collections}
Implementations MUST support binding of the following collection interfaces, classes and their implementations. [JSB-\ref{sec:collections}-1] Implementations of interfaces below MUST provide accessible default constructor.[JSB-\ref{sec:collections}-2] JSON Binding implementations MUST report unmarshalling error if default constructor is not present or is not in accessible scope. [JSB-\ref{sec:collections}-3]

\begin{list}{$\bullet$}{\parsep 0em \labelwidth 0em}
\item java.util.Collection
\item java.util.Map
\item java.util.Set
\item java.util.HashSet
\item java.util.NavigableSet
\item java.util.SortedSet
\item java.util.TreeSet
\item java.util.LinkedHashSet
\item java.util.TreeHashSet
\item java.util.HashMap
\item java.util.NavigableMap
\item java.util.SortedMap
\item java.util.TreeMap
\item java.util.LinkedHashMap
\item java.util.TreeHashMap
\item java.util.List
\item java.util.ArrayList
\item java.util.LinkedList
\item java.util.Deque
\item java.util.ArrayDeque
\item java.util.Queue
\item java.util.PriorityQueue
\item java.util.EnumSet
\item java.util.EnumMap
\end{list}
For interfaces defined above, following table defines default implementation types. Default implementation type for a class, field or property with interface type is the exact type used at runtime to unmarshall JSON values into the field or property. [JSB-\ref{sec:collections}-4]

\begin{tabularx}{\textwidth}{ |X|X| }
\hline
Interface & Default implementation type \\ 
\hline
java.util.Collection & java.util.ArrayList \\
java.util.Set & java.util.HashSet \\
java.util.NavigableSet & java.util.TreeSet \\
java.util.SortedSet & java.util.TreeSet \\
java.util.Map & java.util.HashMap \\
java.util.SortedMap & java.util.TreeMap \\
java.util.NavigableMap & java.util.TreeMap \\
java.util.Deque & java.util.ArrayDeque \\
java.util.Queue & java.util.ArrayDeque \\
\hline
\end{tabularx}

\section{Arrays}
\label{sec:arrays}
JSON Binding implementations MUST support binding of Java arrays of all supported Java types from this chapter into/from JSON array structures as defined in Section 5 of RFC 7159 \cite{rfc7159}. [JSB-\ref{sec:arrays}-1] Arrays of primitive types and multi-dimensional arrays MUST be supported. [JSB-\ref{sec:untyped}-2]

\section{Null value handling}
\label{sec:null}

\subsection{Null Java field}
\label{subsec:nullfield}
The result of marshalling java field with null value is absence of the property in resulting JSON document. [JSB-\ref{subsec:nullfield}-1]
Unmarshalling operation of a property absent in JSON document MUST not set the value of the field, setter (if available) MUST not be called, thus original value of the field MUST be preserved. [JSB-\ref{subsec:nullfield}-2]

\subsection{Null Array Values}
\label{subsec:nullarray}
The result of unmarshalling n-ary array represented in JSON document is n-ary Java array. [JSB-\ref{subsec:nullarray}-1]. Null value in JSON array is represented by null values in Java array. [JSB-\ref{subsec:nullarray}-2]
Marshalling operation on Java array with null value at index i must output null value at index i of the array in resulting JSON document. [JSB-\ref{subsec:nullarray}-3]

\section{Names and identifiers}
\label{sec:naming}
According to RFC 7159 Section 7 \cite{rfc7159}, every Java identifier name can be transformed using identity function into a valid JSON String. Identity function should be used for transforming Java identifier names into name Strings in JSON document. [JSB-\ref{sec:naming}-1]
For unmarshal operations defined in \fullref{sec:untyped} section, identity function is used to transform JSON name strings into Java String instances in the resulting map Map<String, Object>. [JSB-\ref{sec:naming}-2] Identity function is used also for other unmarshalling operations. [JSB-\ref{sec:naming}-3] If a Java identifier with corresponding name does not exist or is not accessible, the implementations MUST report error. [JSB-\ref{sec:naming}-4] Naming and error reporting strategies can be further customized in \fullref{customization}.

\section{Generics}
\label{sec:generics}
JSON Binding implementations MUST support binding of generic types. [JSB-\ref{sec:generics}-1] Due to type erasure, there are situations when it is not possible to obtain generic type information.

There are two ways in JSON Binding how to obtain generic type information. [JSB-\ref{sec:generics}-2] If there is a class file available (in the following text referred as static type information), it is possible to obtain generic type information (effectively generic type declaration) from Signature attribute (if this information is present). [JSB-\ref{sec:generics}-3]The second option is to provide generic type information at runtime. [JSB-\ref{sec:generics}-4] To provide generic type information at runtime, an argument of java.lang.reflect.Type MUST be passed to Jsonb::toJson or to Jsonb::fromJson method. [JSB-\ref{sec:generics}-5]

\subsection{Type resolution algorithm}
\label{sec:type_resolution_algorithm}
There are several distinct kinds of information JSON Binding may have about the given type of field/class/interface: [JSB-\ref{sec:type_resolution_algorithm}-1]

\begin{enumerate}
	\item runtime type provided via java.lang.reflect.Type parameter passed to Jsonb::toJson or Jsonb::fromJson method
	\item static type provided in class file (effectively stored in Signature attribute)
	\item raw type
	\item no information about the type
\end{enumerate}

If there is no information about the type, JSON Binding implementation MUST treat this type as java.lang.Object. [JSB-\ref{sec:type_resolution_algorithm}-2]
If only raw type of given field/class/interface is known, then this type MUST be treated like raw type. [JSB-\ref{sec:type_resolution_algorithm}-3] For example, if there is only information that given field/class/interface is of type java.util.ArrayList, than this type MUST be treated as java.util.ArrayList\(<\)Object\(>\). [JSB-\ref{sec:type_resolution_algorithm}-4]

JSON Binding implementations MUST use the most specific type derived from the information available. [JSB-\ref{sec:type_resolution_algorithm}-5]

Let's consider situation when there is only static type information of given field/class/interface known, and there is no runtime type information. Let's GenericClass\(<T_1,\dotsc,T_n>\) be part of generic type declaration, where GenericClass is name of the generic type and \(T_1,\dotsc,T_n\) are type parameters. For every \(T_i\), where i in \(1,\dotsc,n\), there are 3 possible options: [JSB-\ref{sec:type_resolution_algorithm}-6]

\begin{enumerate}
	\item \(T_i\) is concrete parameter type
	\item \(T_i\) is bounded parameter type
	\item \(T_i\) is wildcard parameter type without bounds
\end{enumerate}

In the first case, the most specific parameter type MUST be given concrete parameter type \(T_i\). [JSB-\ref{sec:type_resolution_algorithm}-7]

For bounded parameter type, let's have bounds \(B_1,\dotsc,B_m\). If m = 1, then the most specific parameter type MUST be derived from the given bound \(B_1\). [JSB-\ref{sec:type_resolution_algorithm}-8] If \(B_1\) is class or interface, the most specific parameter type MUST be given class or interface. [JSB-\ref{sec:type_resolution_algorithm}-9] Otherwise the most specific parameter type SHOULD be java.lang.Object. [JSB-\ref{sec:type_resolution_algorithm}-10]

If multiple bounds are specified, the first step is to resolve every bound separately. Let's define result of such resolution as \(S_1,\dotsc,S_m\) specific parameter types. If \(S_1,\dotsc,S_m\) are java.lang.Object, then the bounded parameter type \(T_i\) MUST be java.lang.Object. [JSB-\ref{sec:type_resolution_algorithm}-11] If there is exactly one \(S_k\), where \(1<=k<=m\) is different than java.lang.Object, then the most specific parameter type for this bounded parameter type \(T_i\) MUST be \(S_k\). [JSB-\ref{sec:type_resolution_algorithm}-12] If exists \(S_{k1},S_{k2}\), where \(1<=k1<=k2<=m\), then the most specific parameter type is \(S_{k1}\). [JSB-\ref{sec:type_resolution_algorithm}-13]

For wildcard parameter type without bounds, then the most specific parameter type MUST be java.lang.Object. [JSB-\ref{sec:type_resolution_algorithm}-14]

Any unresolved type parameter MUST be treated as java.lang.Object. [JSB-\ref{sec:type_resolution_algorithm}-15]

If runtime type is provided via java.lang.reflect.Type parameter passed to Jsonb::toJson or Jsonb::fromJson method, than this type overrides static type declaration wherever applicable. [JSB-\ref{sec:type_resolution_algorithm}-16]

There are situations, when it is necessary to use combination of runtime and static type information. [JSB-\ref{sec:type_resolution_algorithm}-17]

\begin{figure}[hbtp]
\caption{Example Type resolution}
\label{ex1}
\begin{listing}{1}
public class MyGenericType<T,U> {
    public T field1;
    public U field2;
}\end{listing}
\end{figure}

To resolve type of field1, we need runtime type of MyGenericType and static type of field1.
