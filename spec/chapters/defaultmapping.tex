\chapter{Default Mapping}
\label{defaultmapping}

This section defines the default binding (representation) of Java components and classes to JSON documents. The default binding defined here can be further customized as specified in Chapter \ref{customization} - \nameref{customization}.

\section{General}
\label{sec:general}
JSON Binding implementations ('implementations' in further text) MUST support binding of JSON documents as defined in RFC 7159 JSON Grammar \cite{rfc7159}. Serialized JSON output MUST conform to the RFC 7159 JSON Grammar \cite{rfc7159} and be encoded in UTF-8 encoding as defined in Section 8.1 (Character Encoding) of RFC 7159 \cite{rfc7159}. [JSB-\ref{sec:general}-1] 
Implementations MUST support deserialization of documents conforming to RFC 7159 JSON Grammar \cite{rfc7159}. [JSB-\ref{sec:general}-2] In addition, implementations SHOULD NOT allow deserialization of RFC 7159 \cite{rfc7159} non-conforming text (e.g. unsupported encoding, ...) and report error in such cases. [JSB-\ref{sec:general}-3] 
Detection of UTF encoding of a deserialized document MUST follow the encoding process defined in the Section 3 (Encoding) of RFC 4627 \cite{rfc4627}. [JSB-\ref{sec:general}-4] 
Implementations SHOULD ignore the presence of an UTF byte order mark (BOM) and not treat it as an error.[JSB-\ref{sec:general}-5]

\section{Errors}
\label{sec:errors}
Implementations SHOULD NOT allow deserialization of RFC 7159 \cite{rfc7159} non-conforming text (e.g. unsupported encoding, ...) and report an error in such case. [JSB-\ref{sec:errors}-1] 
Implementations SHOULD also report an error during a deserialization operation, if it is not possible to represent a JSON document value with the expected Java type. [JSB-\ref{sec:errors}-2]

\section{Basic Java Types}
\label{sec:basic}
Implementations MUST support binding of the following basic Java classes and their corresponding primitive types: [JSB-\ref{sec:basic}-1]
\begin{list}{$\bullet$}{\parsep 0em \labelwidth 0em}
\item java.lang.String
\item java.lang.Character
\item java.lang.Byte
\item java.lang.Short
\item java.lang.Integer
\item java.lang.Long
\item java.lang.Float
\item java.lang.Double
\item java.lang.Boolean
\end{list}

\subsection{java.lang.String, Character}
\label{subsec:string}
Instances of type java.lang.String and java.lang.Character are serialized to JSON String values as defined within RFC 7159 Section 7 (Strings) \cite{rfc7159} in UTF-8 encoding without a byte order mark. [JSB-\ref{subsec:string}-1] 
Implementations SHOULD support deserialization of JSON text in other (than UTF-8) UTF encodings into java.lang.String instances.[JSB-\ref{subsec:string}-2]

\subsection{java.lang.Byte, Short, Integer, Long, Float, Double}
\label{subsec:number}
Serialization of type java.lang.Byte, Short, Integer, Long, Float or Double (and their corresponding primitive types) to a JSON Number
MUST follow the conversion process defined in the javadoc specification for the corresponding type's toString() method [JSB-\ref{subsec:number}-1]. 
Deserialization of a JSON value into java.lang.Byte, Short, Integer, Long, Float or Double instance (or their corresponding primitive types) 
MUST follow the conversion process defined in the javadoc specification for the corresponding
parse\${Type} method, such as java.lang.Byte.parseByte() for Byte. [JSB-\ref{subsec:number}-2]

\subsection{java.lang.Boolean}
\label{subsec:boolean}
Serialization of type java.lang.Boolean and its corresponding boolean primitive type to a JSON value 
MUST follow the conversion process defined in the javadoc specification for java.lang.Boolean.toString() method [JSB-\ref{subsec:boolean}-1]. 
Deserialization of a JSON value into java.lang.Boolean instance or boolean primitive type
MUST follow the conversion process defined in the javadoc specification for java.lang.Boolean.parseBoolean() method. [JSB-\ref{subsec:boolean}-2]

\subsection{java.lang.Number}
\label{subsec:abstractnumber}
Serialization of java.lang.Number instances (if their more concrete type is not defined elsewhere in this chapter) to a JSON string 
MUST retrieve double value from java.lang.Number.doubleValue() method and convert it to a JSON Number as defined in \fullref{subsec:number}. [JSB-\ref{subsec:abstractnumber}-1]. 
Deserialization of a JSON value into java.lang.Number type MUST return an instance of java.math.BigDecimal by using conversion process defined 
in the javadoc specification for constructor of java.math.BigDecimal with java.lang.String argument. [JSB-\ref{subsec:abstractnumber}-2]

\section{Specific Standard Java SE Types}
\label{sec:specific}
Implementations MUST support binding of the following standard Java SE classes: [JSB-\ref{sec:specific}-1]
\begin{list}{$\bullet$}{\parsep 0em \labelwidth 0em}
\item java.math.BigInteger
\item java.math.BigDecimal
\item java.net.URL
\item java.net.URI
\item java.util.Optional
\item java.util.OptionalInt
\item java.util.OptionalLong
\item java.util.OptionalDouble
\end{list}

\subsection{java.math.BigInteger, BigDecimal}
\label{subsec:bignumber}
Serialization of type java.math.BigInteger or BigDecimal to a JSON Number 
MUST follow the conversion process defined in the javadoc specification for the corresponding type's toString() method [JSB-\ref{subsec:bignumber}-1]. 
Deserialization of a JSON value into java.math.BigInteger or BigDecimal instance 
MUST follow the conversion process defined in the javadoc specification for the constructor of java.math.BigInteger or BigDecimal with java.lang.String argument. [JSB-\ref{subsec:bignumber}-2]

\subsection{java.net.URL, URI}
\label{subsec:url}
Serialization of type java.net.URL or URI to a JSON String 
MUST follow the conversion process defined in the javadoc specification for the corresponding type's toString() method [JSB-\ref{subsec:url}-1]. 
Deserialization of a JSON value into java.net.URL or URI instance 
MUST follow the conversion process defined in the javadoc specification for the constructor of java.net.URL or URI with java.lang.String argument. [JSB-\ref{subsec:url}-2]

\subsection{java.util.Optional, OptionalInt, OptionalLong, OptionalDouble}
\label{subsec:optional}
Non-empty instances of type java.util.Optional, OptionalInt, OptionalLong, OptionalDouble are serialized to a JSON value by retrieving their contained instance and 
converting it to JSON value based on its type and corresponding mapping definitions within this chapter. [JSB-\ref{subsec:optional}-1] 
Empty optional instances serialized as object instance properties are ignored during serialization. [JSB-\ref{subsec:optional}-2] 
Empty optional instances serialized as JSON array elements are serialized as null values [JSB-\ref{subsec:optional}-3]. 
Deserialization into Optional, OptionalInt, OptionalLong, OptionalDouble returns empty optional value for properties which are not present in JSON document or contain a null value. [JSB-\ref{subsec:optional}-4] 
Otherwise any non-empty Optional, OptionalInt, OptionalLong, OptionalDouble value is deserialized based on mappings defined in this chapter.[JSB-\ref{subsec:optional}-5]

Instances of type java.util.Optional\textless T\textgreater\space are serialized to a JSON value as JSON objects when T alone would be serialized as JSON object. 
When T would be serialized as a JSON value (e.g. java.lang.String, java.lang.Integer), an instance of java.util.Optional\textless T\textgreater\space is serialized as a JSON value (without curly brackets). [JSB-\ref{subsec:optional}-6]

Deserialization of a JSON value into java.util.Optional\textless T\textgreater\space MUST be supported if deserialization of a JSON value into instance of T is supported. [JSB-\ref{subsec:optional}-7]

\section{Dates}
\label{sec:dates}
Implementations MUST support binding of the following standard Java date/time classes: 
[JSB-\ref{sec:dates}-1]
\begin{list}{$\bullet$}{\parsep 0em \labelwidth 0em}
\item java.util.Date
\item java.util.Calendar
\item java.util.GregorianCalendar
\item java.util.TimeZone
\item java.util.SimpleTimeZone
\item java.time.Instant
\item java.time.Duration
\item java.time.Period
\item java.time.LocalDate
\item java.time.LocalTime
\item java.time.LocalDateTime
\item java.time.ZonedDateTime
\item java.time.ZoneId
\item java.time.ZoneOffset
\item java.time.OffsetDateTime
\item java.time.OffsetTime
\end{list}

If not specified otherwise in this section, GMT standard time zone and offset specified from UTC Greenwich is used. [JSB-\ref{sec:dates}-2]
If not specified otherwise, the date time format for serialization and deserialization is ISO 8601 without offset, as specified in java.time.format.DateTimeFormatter.ISO\verb|_|DATE. [JSB-\ref{sec:dates}-3]
Implementations MUST report an error if the datetime string in a JSON document does not correspond to the expected datetime format. [JSB-\ref{sec:dates}-4]

\subsection{java.util.Date, Calendar, GregorianCalendar}
\label{subsec:datecalendar}
The serialization format of java.util.Date, Calendar, GregorianCalendar instances with no time information is ISO\verb|_|DATE. [JSB-\ref{subsec:datecalendar}-1]. If time information is present, the format is ISO\verb|_|DATE\verb|_|TIME [JSB-\ref{subsec:datecalendar}-2].

Implementations MUST support deserialization of both ISO\verb|_|DATE and ISO\verb|_|DATE\verb|_|TIME into java.util.Date, Calendar, GregorianCalendar instances. [JSB-\ref{subsec:datecalendar}-3]

\subsection{java.util.TimeZone, SimpleTimeZone}
\label{subsec:dateTimezone}

Implementations MUST support deserialization of any time zone format specified in java.util.TimeZone into a field or property of type java.util.TimeZone, SimpleTimeZone. [JSB-\ref{subsec:dateTimezone}-1] 
Implementations MUST report an error for deprecated three-letter time zone IDs as specified in java.util.Timezone. [JSB-\ref{subsec:dateTimezone}-2]
The serialization format of java.util.TimeZone and SimpleTimeZone is NormalizedCustomID as specified in java.util.TimeZone. [JSB-\ref{subsec:dateTimezone}-3]

\subsection{java.time.*}
\label{subsec:time}

The serialization output for a java.time.Instant instance MUST be in a ISO\verb|_|INSTANT format, as specified in 
java.time.format.DateTimeFormatter. [JSB-\ref{subsec:time}-1]
Implementations MUST support the deserialization of an ISO\verb|_|INSTANT formatted JSON string to a java.time.Instant instance. [JSB-\ref{subsec:time}-2]

For other java.time.* classes, the following mapping table maps Java types to their corresponding formats: [JSB-\ref{subsec:time}-3]

\begin{tabularx}{\textwidth}{ |X|X| }
\hline
Java Type & Format \\ 
\hline
java.time.Instant & ISO\verb|_|INSTANT\\
java.time.LocalDate & ISO\verb|_|LOCAL\verb|_|DATE\\
java.time.LocalTime & ISO\verb|_|LOCAL\verb|_|TIME\\
java.time.LocalDateTime & ISO\verb|_|LOCAL\verb|_|DATE\verb|_|TIME\\
java.time.ZonedDateTime & ISO\verb|_|ZONED\verb|_|DATE\verb|_|TIME\\
java.time.OffsetDateTime & ISO\verb|_|OFFSET\verb|_|DATE\verb|_|TIME\\
java.time.OffsetTime & ISO\verb|_|OFFSET\verb|_|TIME\\
\hline
\end{tabularx}

Implementations MUST support the deserialization of any time zone ID format specified in java.time.ZoneId into a field or property of type java.time.ZoneId. [JSB-\ref{subsec:time}-4]
The serialization format of java.time.ZoneId is the normalized zone ID as specified in java.time.ZoneId. [JSB-\ref{subsec:time}-5]

Implementations MUST support the deserialization of any time zone ID format specified in java.time.ZoneOffset into a field or property of type java.time.ZoneOffset. [JSB-\ref{subsec:time}-6]
The serialization format of java.time.ZoneOffset is the normalized zone ID as specified in java.time.ZoneOffset. [JSB-\ref{subsec:time}-7]

Implementations MUST support the deserialization of any duration format specified in java.time.Duration into a field or property of type java.time.Duration. [JSB-\ref{subsec:time}-8] This is super-set of ISO 8601 duration format.
The serialization format of java.time.Duration is the ISO 8601 seconds based representation, such as
\verb|PT8H6M12.345S|. [JSB-\ref{subsec:time}-9]

Implementations MUST support the deserialization of any period format specified in java.time.Period into a field or property of type java.time.Period. [JSB-\ref{subsec:time}-10] 
This is a super-set of ISO 8601 period format.
The serialization format of java.time.Period is ISO 8601 period representation. [JSB-\ref{subsec:time}-11] 
A zero-length period is represented as zero days \verb|'P0D'|. [JSB-\ref{subsec:time}-12]


\section{Untyped mapping}
\label{sec:untyped}
For an unspecified output type of a deserialization operation, as well as where output type is specified as Object.class, implementations MUST deserialize a JSON document using Java runtime types specified in table below: [JSB-\ref{sec:untyped}-1]

\begin{tabularx}{\textwidth}{ |X|X| }
\hline
JSON value & Java type \\ 
\hline
object & java.util.Map\textless String,Object\textgreater\\
array & java.util.List\textless Object\textgreater \\
string & java.lang.String \\
number & java.lang.Integer|Long|java.math.BigDecimal \\
true, false & java.lang.Boolean \\
null & null \\
\hline
\end{tabularx}

JSON object values are deserialized into an implementation of java.util.Map\textless String, Object\textgreater\space with a predictable iteration order.[JSB-\ref{sec:untyped}-2]

JSON number values are deserialized into the smallest of types Integer, Long, BigDecimal that can hold the value represented by number without loss of value or precision.[JSB-\ref{sec:untyped}-3]

\section{Java Class}
\label{sec:class}
Any instance passed to a deserialization operation must have a public or protected no-argument constructor. Implementations SHOULD throw an error if this condition is not met. [JSB-\ref{sec:class}-1] This limitation does not apply to serialization operations, as well as to classes which specify explicit instantiation methods as described in \fullref{sec:custom_instantiation} [JSB-\ref{sec:class}-2]

\subsection{Scope and Field access strategy}
\label{subsec:fieldstrategy}
For a deserialization operation of a Java property, if a matching public setter method exists, the method is called to set the value of the property. 
If a matching setter method with private, protected, or defaulted to package-only access exists, then this field is ignored. 
If no matching setter method exists and the field is public, then direct field assignment is used. [JSB-\ref{subsec:fieldstrategy}-1]

For a serialization operation, if a matching public getter method exists, the method is called to obtain the value of the property. 
If a matching getter method with private, protected, or defaulted to package-only access exists, then this field is ignored. 
If no matching getter method exists and the field is public, then the value is obtained directly from the field. [JSB-\ref{subsec:fieldstrategy}-2]

JSON Binding implementations MUST NOT deserialize into transient, final or static fields and MUST ignore name/value pairs corresponding to such fields. [JSB-\ref{subsec:fieldstrategy}-3]

Implementations MUST support serialization of final fields. [JSB-\ref{subsec:fieldstrategy}-4] Transient and static fields MUST be ignored during serialization operation. [JSB-\ref{subsec:fieldstrategy}-5]

If a JSON document contains a name/value pair not corresponding to field or setter method, then this name/value pair MUST be ignored. [JSB-\ref{subsec:fieldstrategy}-6]

Public getter/setter methods without a corresponding field MUST be supported. 
When only public getter/setter method without corresponding field is present in the class, the getter method is called to obtain the value to serialize, and the setter method is called during deserialization operation. [JSB-\ref{subsec:fieldstrategy}-7]

\subsection{Nested Classes}
\label{subsec:nestedclass}
Implementations MUST support the binding of public and protected nested classes. [JSB-\ref{subsec:nestedclass}-1] 
For deserialization operations, both nested and encapsulating classes MUST fulfill the same instantiation requirements as specified in \fullref{subsec:fieldstrategy}. [JSB-\ref{subsec:nestedclass}-2]

\subsection{Static Nested Classes}
\label{subsec:staticnested}
Implementations MUST support the binding of public and protected static nested classes. [JSB-\ref{subsec:staticnested}-1] 
For deserialization operations, the nested class MUST fulfill the same instantiation requirements as specified in \fullref{subsec:fieldstrategy}. [JSB-\ref{subsec:staticnested}-2]

\subsection{Anonymous Classes}
\label{subsec:anonymous}
Deserialization into anonymous classes is not supported. Serialization of anonymous classes is supported by default object mapping. [JSB-\ref{subsec:nestedclass}-1]

\section{Polymorphic Types}
\label{sec:polymorph}
Deserialization into polymorphic types is not supported by default mapping. [JSB-\ref{sec:polymorph}-1]

\section{Enum}
\label{sec:enum}
Serialization of an Enum instance to a JSON String value 
MUST follow the conversion process defined in javadoc specification for their toString method [JSB-\ref{sec:enum}-1]. 
Deserialization of a JSON value into an enum instance MUST be done by calling the enum's valueOf(String) method. [JSB-\ref{sec:enum}-2]

\section{Interfaces}
\label{sec:interfaces}
Implementations MUST support the deserialization of specific interfaces defined in  \fullref{sec:collections}, and \fullref{subsec:abstractnumber}. [JSB-\ref{sec:interfaces}-1] 
Deserialization to other interfaces is not supported and implementations SHOULD report error in such case. [JSB-\ref{sec:interfaces}-2] 
If a class property is defined with an interface and not concrete type, then the mapping for a serialized property is resolved based on its runtime type.[JSB-\ref{sec:interfaces}-3]

\section{Collections}
\label{sec:collections}
Implementations MUST support the binding of the following collection interfaces, classes and their implementations. [JSB-\ref{sec:collections}-1] 
Implementations of these interfaces must provide an accessible default constructor. JSON Binding implementations MUST report a deserialization error if a default constructor is not present or is not in accessible scope. [JSB-\ref{sec:collections}-2]

\begin{list}{$\bullet$}{\parsep 0em \labelwidth 0em}
\item java.util.Collection
\item java.util.Map
\item java.util.Set
\item java.util.HashSet
\item java.util.NavigableSet
\item java.util.SortedSet
\item java.util.TreeSet
\item java.util.LinkedHashSet
\item java.util.TreeHashSet
\item java.util.HashMap
\item java.util.NavigableMap
\item java.util.SortedMap
\item java.util.TreeMap
\item java.util.LinkedHashMap
\item java.util.TreeHashMap
\item java.util.List
\item java.util.ArrayList
\item java.util.LinkedList
\item java.util.Deque
\item java.util.ArrayDeque
\item java.util.Queue
\item java.util.PriorityQueue
\item java.util.EnumSet
\item java.util.EnumMap
\end{list}

\section{Arrays}
\label{sec:arrays}
JSON Binding implementations MUST support the binding of Java arrays of all supported Java types from this chapter into/from JSON array structures as defined in Section 5 of RFC 7159 \cite{rfc7159}. [JSB-\ref{sec:arrays}-1] Arrays of primitive types and multi-dimensional arrays MUST be supported. [JSB-\ref{sec:untyped}-2]

\section{Attribute order}
\label{sec:attributes}

Declared fields MUST be serialized in lexicographical order into the resulting JSON document. In case of inheritance, declared fields of super class MUST be serialized before declared fields of child class. [JSB-\ref{sec:attributes}-1]

When deserializing a JSON document, declared fields MUST be set in the order of attributes present in the JSON document. [JSB-\ref{sec:attributes}-2]

\section{Null value handling}
\label{sec:null}

\subsection{Null Java field}
\label{subsec:nullfield}
The result of serializing a java field with a null value is the absence of the property in the resulting JSON document. [JSB-\ref{subsec:nullfield}-1]
The deserialization operation of a property absent in JSON document MUST not set the value of the field, the setter (if available) MUST not be called, and thus original value of the field MUST be preserved. [JSB-\ref{subsec:nullfield}-2]

The deserialization operation of a property with a null value in a JSON document MUST set the value of the field to null value (or call setter with null value if setter is present). [JSB-\ref{subsec:nullfield}-3]

\subsection{Null Array Values}
\label{subsec:nullarray}
The result of deserialization n-ary array represented in JSON document is n-ary Java array. [JSB-\ref{subsec:nullarray}-1]. Null value in JSON array is represented by null value in Java array. [JSB-\ref{subsec:nullarray}-2]
Serialization operation on Java array with null value at index i MUST output null value at index i of the array in resulting JSON document. [JSB-\ref{subsec:nullarray}-3]

\section{Names and identifiers}
\label{sec:naming}
According to RFC 7159 Section 7 \cite{rfc7159}, every Java identifier name can be transformed using identity function into a valid JSON String. Identity function MUST be used for transforming Java identifier names into Strings in JSON document. [JSB-\ref{sec:naming}-1]
For deserialization operations defined in \fullref{sec:untyped} section, identity function is used to transform JSON name strings into Java String instances in the resulting map Map\textless String, Object\textgreater. [JSB-\ref{sec:naming}-2] If a Java identifier with corresponding name does not exist or is not accessible, the implementations MUST report error. [JSB-\ref{sec:naming}-3] Naming and error reporting strategies can be further customized in \fullref{customization}.

\section{Big numbers}
\label{sec:big_numbers}
JSON Binding implementation MUST serialize numbers that express greater magnitude or precision than an IEEE 754 double precision number as strings. [JSB-\ref{sec:big_numbers}-1]

\section{Generics}
\label{sec:generics}
JSON Binding implementations MUST support binding of generic types. [JSB-\ref{sec:generics}-1] Due to type erasure, there are situations when it is not possible to obtain generic type information.

There are two ways for JSON Binding implementations to obtain generic type information. If there is a class file available (in the following text referred as static type information), it is possible to obtain generic type information (effectively generic type declaration) from Signature attribute (if this information is present). [JSB-\ref{sec:generics}-2] The second option is to provide generic type information at runtime. To provide generic type information at runtime, an argument of java.lang.reflect.Type MUST be passed to Jsonb::toJson or to Jsonb::fromJson method. [JSB-\ref{sec:generics}-3]

\subsection{Type resolution algorithm}
\label{sec:type_resolution_algorithm}
There are several levels of information JSON Binding implementations may obtain about the type of field/class/interface: [JSB-\ref{sec:type_resolution_algorithm}-1]

\begin{enumerate}
	\item runtime type provided via java.lang.reflect.Type parameter passed to Jsonb::toJson or Jsonb::fromJson method
	\item static type provided in class file (effectively stored in Signature attribute)
	\item raw type
	\item no information about the type
\end{enumerate}

If there is no information about the type, JSON Binding implementation MUST treat this type as java.lang.Object. [JSB-\ref{sec:type_resolution_algorithm}-2]
If only raw type of given field/class/interface is known, then the type MUST be treated like raw type. [JSB-\ref{sec:type_resolution_algorithm}-3] For example, if the only available information is that given field/class/interface is of type java.util.ArrayList, than the type MUST be treated as java.util.ArrayList\(<\)Object\(>\).

JSON Binding implementations MUST use the most specific type derived from the information available. [JSB-\ref{sec:type_resolution_algorithm}-4]

Let's consider situation when there is only a static type information of a given field/class/interface known, and there is no runtime type information available. Let GenericClass\(<T_1,\dotsc,T_n>\) be part of generic type declaration, where GenericClass is name of the generic type and \(T_1,\dotsc,T_n\) are type parameters. For every \(T_i\), where i in \(1,\dotsc,n\), there are 3 possible options: [JSB-\ref{sec:type_resolution_algorithm}-5]

\begin{enumerate}
	\item \(T_i\) is concrete parameter type
	\item \(T_i\) is bounded parameter type
	\item \(T_i\) is wildcard parameter type without bounds
\end{enumerate}

In case 1, the most specific parameter type MUST be given concrete parameter type \(T_i\). [JSB-\ref{sec:type_resolution_algorithm}-6]

For bounded parameter type, let's use bounds \(B_1,\dotsc,B_m\). If m = 1, then the most specific parameter type MUST be derived from the given bound \(B_1\). [JSB-\ref{sec:type_resolution_algorithm}-7] If \(B_1\) is class or interface, the most specific parameter type MUST be the class or interface. [JSB-\ref{sec:type_resolution_algorithm}-8] Otherwise, the most specific parameter type SHOULD be java.lang.Object. [JSB-\ref{sec:type_resolution_algorithm}-9]

If multiple bounds are specified, the first step is to resolve every bound separately. Let's define result of such resolution as \(S_1,\dotsc,S_m\) specific parameter types. If \(S_1,\dotsc,S_m\) are java.lang.Object, then the bounded parameter type \(T_i\) MUST be java.lang.Object. [JSB-\ref{sec:type_resolution_algorithm}-10] 
If there is exactly one \(S_k\), where \(1<=k<=m\) is different than java.lang.Object, then the most specific parameter type for this bounded parameter type \(T_i\) MUST be \(S_k\). [JSB-\ref{sec:type_resolution_algorithm}-11] If there exists \(S_{k1},S_{k2}\), where \(1<=k1<=k2<=m\), then the most specific parameter type is \(S_{k1}\). [JSB-\ref{sec:type_resolution_algorithm}-12]

For wildcard parameter type without bounds, the most specific parameter type MUST be java.lang.Object. [JSB-\ref{sec:type_resolution_algorithm}-13]

Any unresolved type parameter MUST be treated as java.lang.Object. [JSB-\ref{sec:type_resolution_algorithm}-14]

If runtime type is provided via java.lang.reflect.Type parameter passed to Jsonb::toJson or Jsonb::fromJson method, than that runtime type overrides static type declaration wherever applicable. [JSB-\ref{sec:type_resolution_algorithm}-15]

There are situations when it is necessary to use combination of runtime and static type information. 
\begin{figure}[hbtp]
\caption{Example Type resolution}
\label{ex2}
\begin{listing}{1}
public class MyGenericType<T,U> {
    public T field1;
    public U field2;
}\end{listing}
\end{figure}

To resolve type of field1, runtime type of MyGenericType and static type of field1 is required.

\section{Must-Ignore policy}
\label{sec:must_ignore_policy}
When JSON Binding implementation during deserialization encounters key in key/value pair that it does not recognize, it should treat the rest of the JSON document as if the element simply did not appear, and in particular, the implementation MUST NOT treat this as an error condition. [JSB-\ref{sec:must_ignore_policy}-1]

\section{Uniqueness of properties}
\label{sec:uniqueness_properties}
JSON Binding implementations MUST NOT produce JSON documents with members with duplicate names. In this context, "duplicate" means that the names, after processing any escaped characters, are identical sequences of Unicode characters. [JSB-\ref{sec:uniqueness_properties}-1]

When non-unique property (after override and rename) is found, implementation MUST throw an exception. This doesn't apply for customized user serialization behavior implemented with the usage of JsonbAdapter mechanism. [JSB-\ref{sec:uniqueness_properties}-2]

\section{JSON Processing integration}
\label{sec:jsonp_integration}
JSON Binding implementations MUST support binding of the following JSON Processing types. [JSB-\ref{sec:jsonp_integration}-1]

\begin{list}{$\bullet$}{\parsep 0em \labelwidth 0em}
\item javax.json.JsonObject
\item javax.json.JsonArray
\item javax.json.JsonStructure
\item javax.json.JsonValue
\item javax.json.JsonPointer
\item javax.json.JsonString
\item javax.json.JsonNumber
\end{list}

Serialization of supported javax.json.* objects/interfaces/fields MUST have the same result as serialization these objects with javax.json.JsonWriter. [JSB-\ref{sec:jsonp_integration}-2]
Deserialization into supported javax.json.* objects/interfaces/fields MUST have the same result as deserialization into such objects with javax.json.JsonReader. [JSB-\ref{sec:jsonp_integration}-3]